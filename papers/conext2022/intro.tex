\subsection{Background}

%% cloud and virtualization technologies
Cloud computing was brought about by virtualization technologies,
enabling to build large pools of virtualized computing resources (e.g.,
servers, storage, network) and dynamically allocate necessary resources
for providing services.
However, cloud computing system itself is still on physical cloud
infrastructure, that is, cloud system operators manage phisical
servers and network switches to meet the overall demands without
overloading certain resources.

Next challenge would be to virtualize cloud infrastructure!
By decoupling infrastructure management from service management,
cloud operators can be liberated from hardware resource management.
This is similar to what happend to system administrators by cloud
computing.

%% Edge computing and micro datacenters
We also consider two recent trends in cloud computing.
One trend is edge computing and micro datacenters.
diverse computing resources, geographically scattered around.
it will require a new management mechanism to efficiently utilize such
resources.
Computing resources will be abundant in the environment so that it
will be no longer necessary to micro-manage computing resources.

%% serverless, micro services
Another trend is micro services, or serverless computing.
lightweight services are short lived, which would make resource
allocation much simpler. It would be also possible to migrate service
instances. 

%% energy saving??

\subsection{Cloud Morphing Vision}

Cloud Morphing is our vision for cloud virtualization in the future:
dynamically morphing clouds.
By dynamicaly allocatating micro services over distributed
heterogeneous resources, a cloud service instance emerges at the best
location and, as the usage pattern changes, the service instance also
transforms the locations of the resources and their connections.

For examples, an interactive task will follow the user when the user 
moves to a new location, while a data intensive task will stay close
to the data regardless of the user location. As a result, services are
inherently fault-tolerant and resilient against outages or disasters
since jobs on faulty resources are automatically relocated.

It would require technical advances in many fields to realize such
systems.
Among other things, new resource management model is needed.

%%It is a vision for dynamic distributed resource management in 10 years.
%%research vision to set long-term directions, find/prioritize research
%%agenda (beyond current 5G/MEC, including technologies: networking,
%%cloud, IoT, virtualization, dynamic optimization, resource allocation) .
%%It combines recent technical trends: micro-services/serverless, NFV,
%%hybrid-cloud, (mobile) edge computing, machine learning.

%%Current (edge) cloud infrastructure still designed by human, but to
%%shift to dynamic allocation of storage/computing/networking by machine.
%%assuming it is possible to freely utilize available resources on the
%%net with ``pay-as-you-go'' model.

\subsection{Resource Management Model}

%% resource allocation problems are NP-hard
resource allocation in distributed clouds: non-trivial optimization
problem.

%% even harder for distributed heterogeneous resources
distributed resource management: diverse resources are owned and
managed by different parties.
It requires loose management of resources, as small parties cannot
afford dedicated skilled operators.
The utilization of each resource needs to be easily manipulated,
without affecting the stability of the system.

%% our idea: use congestion pricing mechanism for automatic load management
To this end, we employ dynamic pricing for decentralized resource
allocation whcih works as backpressure against congestion.
By design, serious congestion never happens in the system.

pseudo pricing for manipullating resource allocation.
and a pseudo price is not an actual manetary charge.

There exist a large body of literature formulating resource allocation
as optimization problems.
In this paper, we use terms and tools borrowed from optimization
theory. However, our goal is not to pursue theoretical optimum
allocations, but to present a practical feedback control model for
distributed resource management.


