Congestion pricing and smart data pricing for networking have been
extensively studied~\cite{Sen-2013}.
Early work includes congestion pricing by
Murphy {\em et al.}~\cite{MURPHY19941053},
and auction-based resource allocation by
MacKie-Mason {\em et al.}~\cite{pricing-internet-1994}, both in 1994.
Kelly {\em et al.}~\cite{Kelly-1998} framed congestion pricing in an
optimization framework for fair allocation, which inspired a large
number of the following
work~\cite{Sen-2013,gibbens1999resource,Henderson2001,Briscoe2003-M3I}.
Congestion pricing are also applied to cloud
computing~\cite{Wang-hotcloud2010,Song-2014,Kilcioglu-SIGMETRICS2015,Song-INFOCOM2017}.

Resource allocation for edge computing and micro data centers also has
a rich collection of work; most approaches are based on some form of
optimization and many employ auction models.
The topics are ranging from VM auctions in
IaaS~\cite{Zhang2017-VMauction,Zaman-2013},
load balancing within a data center~\cite{Rikhtegar2021BiTEAD,Chen-SOCC-2014},
to
distributed resource management for microservices~\cite{Suresh-SOA-SOCC2017}.

Xu {\em et al.}~\cite{Xu2017-zenith} proposed the edge computing
infrastructure layer similar to ours, separating the cloud
infrastructure service from the cloud service for micro datacenters,
but an auction model is proposed for resource allocation.

Exponential cost growth as a function of load is well known in packet
switching networks (e.g., M/M/1 queue and CSMA/CD Ethernet).
The idea to use such cost functions for resource management
was in \cite{MURPHY19941053} where
Murphy {\em et al.} used a cost function for distributed bandwidth
allocation in ATM networks in 1994.
Their cost function is a barrier function for utility optimization
and their simple cost minimizing allocation algorithm is also somewhat
similar to our allocation model.

A system model similar to ours is found in~\cite{Wagner-2012} where
Wagner {\em et al.} used congestion pricing for resilient job
allocation in a distributed military cloud.  They used a
game-theoretic resource allocation method based on Nash Bargaining,
and developed a Hadoop-based prototype system.

We were inspired by the concept of micro cloud services to apply packet
switching techniques to distributed heterogeneous clouds, and have
revisited congestion pricing.
To the best of our knowledge, our work is unique in using a convex
cost function for idle-resource pooling.

%%data migration\cite{Pu2015-geodistdata}.
%%distributed data store\cite{Shima2012,Tahoe-2008}.


